\documentclass[11pt, oneside]{article}    % use "amsart" instead of "article" for AMSLaTeX format
\usepackage{geometry}                     % See geometry.pdf to learn the layout options. There are lots.
\geometry{letterpaper}                       % ... or a4paper or a5paper or ...
%\usepackage[parfill]{parskip}         % Activate to begin paragraphs with an empty line rather than an indent
\usepackage{graphicx}            % Use pdf, png, jpg, or epsß with pdflatex; use eps in DVI mode
\usepackage{fullpage}
\usepackage{amssymb}
\usepackage{amsthm}
\usepackage{algorithmicx}
\usepackage{algorithm}
\usepackage[noend]{algpseudocode}
\algrenewcommand\algorithmicprocedure{}
\newtheorem{theorem}{Theorem}
\newtheorem{proposition}{Proposition}
\newtheorem{lemma}{Lemma}

\newcommand{\hw}[2]{\noindent {\bf COMP 4030/6030}: Assignment #2\\
{\bf Due date}: #2\\}

\begin{document}
\hw{1}{03/02/2016}

Email programming solutions to the TA (Quang Tran, qmtran@memphis.edu).  Put ``COMP 4030/6030 assignment 3" on the subject line.
\begin{enumerate}
	\item (Programming assignment, 25 points) Write a Python function called Split, which takes a list of numbers A and returns a list of numbers B with the following properties:
	\begin{itemize}
		\item The numbers in B are simply an rearrangement of the numbers in A.
		\item In particular, A[0] appears somewhere in B, say at index i.  (This means A[0] == B[i])
		\item If j $<=$ i, then B[j] $<=$ B[i].
		\item If j $>$ i, then B[j] $>$ B[i].
	\end{itemize}

Example: Split([{\bf 10},7,20,3,0,9,15,7,18,50]) returns [7,3,0,9,7,{\bf 10},20,15,18,50].  Notice that numbers in the returned list are the same as those in the input list; they are simply rearranged.  Further, the arrangement is such as in the returned list, (i) everything to the left of 10 (A[0]) is less than or equal to it and (ii) everything to the right of 10 is greater than it.
	
	\item (Bonus Programming assignment, 20 points) Rewrite Python program in question 2 to take $\Theta(1)$ of space.  Notice that the space complexity does not account for the input list A.
		
	\item (25 points) Analyze the running time complexity and space complexity of your own program in question 1.

	\item (25 points) Explain why $n^5 + 2n^2 \in \Theta(n^5)$.

	\item (25 points) Given the following function:
\begin{verbatim}
def find_cats(L, idx, output):
   if idx < len(L):
      if "cat" in L[idx]:
         output.append(L[idx])
         find_cats(L, idx+1, output)
      else:
         find_cats(L, idx+1, output)
	\end{verbatim}
Assuming L is a list of strings, explain why after executing the code below, A will be a list of all strings (in L) that contain ``cat". 
\begin{verbatim}
A = []
find_cats(L, 0, A)
\end{verbatim}
 Hint: this is an explanation of $why$ a function is correct, not $how$ it works.  Clearly, in your explanation of $why$, you will probably refer to the logic of the function (i.e. $how$ it works in some places).  But explaining $why$ something is correct and $how$ it works are not the same thing.
\end{enumerate}

\end{document}