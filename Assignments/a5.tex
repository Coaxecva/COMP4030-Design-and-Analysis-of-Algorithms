\documentclass[11pt, oneside]{article}    % use "amsart" instead of "article" for AMSLaTeX format
\usepackage{geometry}                     % See geometry.pdf to learn the layout options. There are lots.
\geometry{letterpaper}                       % ... or a4paper or a5paper or ...
%\usepackage[parfill]{parskip}         % Activate to begin paragraphs with an empty line rather than an indent
\usepackage{graphicx}            % Use pdf, png, jpg, or epsß with pdflatex; use eps in DVI mode
\usepackage{fullpage}
\usepackage{amssymb}
\usepackage{amsthm}
\usepackage{algorithmicx}
\usepackage{algorithm}
\usepackage[noend]{algpseudocode}
\algrenewcommand\algorithmicprocedure{}
\newtheorem{theorem}{Theorem}
\newtheorem{proposition}{Proposition}
\newtheorem{lemma}{Lemma}

\newcommand{\hw}[2]{\noindent {\bf COMP 4030/6030}: Assignment #1\\
{\bf Due date}: #2\\}

\begin{document}
\hw{5}{03/28/2016}

Email programming solutions to the TA (Quang Tran, qmtran@memphis.edu).  Put ``COMP 4030/6030 assignment 5" on the subject line.
\begin{enumerate}
	\item Programming assignment: in this problem, you need to implement different strategies to determine the majority item in a list of items.  There's a special constraint that you can not compare items using less than or greater than operators.  You can only compare if two items are the same.  An item is a majority element if its frequency is greater than ${n \over 2}$, where $n$ is the number of items in the list.  For example, in the list [C, C, T, T, C, T, C, C], C is the majority item, because the frequency of C is 5, which is greater than $4 = {8 \over 2}$.  Another example, the list [C, C, T, C, C, S, T, T] has no majority item.
		\begin{enumerate}
		\item (20 points) Use a brute-force strategy to determine the majority item in $O(n^2)$ time.  Hint: compute the frequency of each item and determine the majority if there is one.
		\item (20 points) Use a divide-and-conquer strategy to determine the majority item in $O(n \log n)$ time.  Hint: 
			\begin{itemize}
			\item Majority(L, i, j) returns the majority element for the interval [i, j] in L.
			\item Split the problem of size $n$ into two subproblems of size ${n \over 2}$ (left half and right half).
			\item After the majority elements of the left half and right half are determined, you will want to compute the majority element for the entire interval [i,j] in O(n) time.				
			\end{itemize}
		\end{enumerate}
	\item (20 points) Use the Master's theorem to find the complexity (in terms of $\Theta$) for the following equation: $T(n) = 2n + 9T({n \over 3})$.  Assuming $T(1) = 1$.
	\item (20 points) Use the Master's theorem to find the complexity (in terms of $\Theta$) for the following equation: $T(n) = 4n^2 + 9T({n \over 3})$.  Assuming $T(1) = 1$.
	\item (20 points) Use the Master's theorem to find the complexity (in terms of $\Theta$) for the following equation: $T(n) = 4n + 4T({n \over 3})$.  Assuming $T(1) = 1$.
\end{enumerate}

\end{document}