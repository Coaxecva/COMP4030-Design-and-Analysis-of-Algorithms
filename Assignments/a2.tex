\documentclass[11pt, oneside]{article}    % use "amsart" instead of "article" for AMSLaTeX format
\usepackage{geometry}                     % See geometry.pdf to learn the layout options. There are lots.
\geometry{letterpaper}                       % ... or a4paper or a5paper or ...
%\usepackage[parfill]{parskip}         % Activate to begin paragraphs with an empty line rather than an indent
\usepackage{graphicx}            % Use pdf, png, jpg, or epsß with pdflatex; use eps in DVI mode
\usepackage{fullpage}
\usepackage{amssymb}
\usepackage{amsthm}
\usepackage{algorithmicx}
\usepackage{algorithm}
\usepackage[noend]{algpseudocode}
\algrenewcommand\algorithmicprocedure{}
\newtheorem{theorem}{Theorem}
\newtheorem{proposition}{Proposition}
\newtheorem{lemma}{Lemma}

\newcommand{\hw}[2]{\noindent {\bf COMP 4030/6030}: Assignment #2\\
{\bf Due date}: #2\\}

\begin{document}
\hw{1}{02/14/2016}

Email programming solutions to the TA (Quang Tran, qmtran@memphis.edu).  Put ``COMP 4030/6030 assignment 1" on the subject line.
\begin{enumerate}
	\item (Programming assignment) Write an {\bf iterative} Python function that takes two sorted lists and returns a sorted list that is a merge of the two lists.  The function looks like this:
\begin{verbatim}
def merge(A, B):
   C = []
   # do some processing and manipulation
   return C
\end{verbatim}
Example, merge([1,3,5,7,9], [0,2,4]) returns [0,1,2,3,4,5,7,9]. 

	\item (Programming assignment) Write a {\bf recursive} Python function that takes two sorted lists and returns a sorted list that is a merge of the two lists.
	
	\item Explain why (i) $10n + 2n^2 \in O(n^2)$ and (ii) $10n + 2n^2 \in \Omega(n^2)$
	\item What is the running time complexity of the following function:

\begin{verbatim}
# L is a list of numbers
def foo(L):
    sum = 0
    for x in L:
        j = 1
        while j < len(L):
            sum += x*x*j
            j = j * 2
    return sum
\end{verbatim}

	\item Explain why the following function correctly adds all items in the input list $L$.
\begin{verbatim}
def add(L):
    if len(L) == 0:
        return 0
    return L[0] + add(L[1:])
\end{verbatim}

	\item Explain why the following function correctly select all even items in the input list $L$.
\begin{verbatim}
def select(L):
    if len(L) == 0:
        return []
    if L[0] % 2 == 0:
        return [L[0]] + select(L[1:])
    return select(L[1:])
\end{verbatim}

\end{enumerate}

\end{document}